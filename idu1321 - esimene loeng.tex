\documentclass{beamer}
\usetheme{metropolis}

\usepackage[utf8]{inputenc}
\usepackage[normalem]{ulem}
\usepackage{csquotes}

\title{IDU1321. Ettevõtte äriarhitektuur}
\subtitle{Esimene loeng}
%\date{10.09.2017}
\author{Andres Kütt}
\institute{Cybernetica, arhitekt}


\begin{document}

\begin{frame}
\titlepage
\end{frame}

\section{Taust ja sissejuhatus}
\begin{frame}{Lektor}

	\begin{description}
		\item[Nimi] Andres Kütt
		\item[Haridus] MsC (Tartu Ülikool), MBA (EBS), MsC (MIT)
		\item[Kogemus] 15 aastat kogemust arhitektuuriga,\\ kokku 24 aastat tarkvaraäris 
		\item[Kohad] Riigiasutused, pangad, Skype, konsultatsioon
		\item[Huvid] Arhitetkuur, keerukus, süsteemidünaamika
	\end{description}
\end{frame}

\begin{frame}{Aine}
	\begin{description}
		\item[Nimetus] IDU1321. Ettevõtte äriarhitektuur
		\item[Eesmärk] Anda kuulajaile nii palju praktiliselt kasutatavat EA-teadmist, kui jõuab
		\item[Meetod] Loengud segatud praktikumiga, iseseisev töö
	\end{description}
\end{frame}

\begin{frame}{Loengutest}
	\begin{itemize}
		\item Tegu on pedagoogilise eksperimendiga, ma ei ole loenguid ja praktikume sel viisil veel seganud
		\item Eksami ja projektiettekannete aeg ja formaat täpsustub
		\item Kodukord
		\begin{itemize}
			\item Kohal olema ei pea
			\item Kui oled kohal, ole kohal
			\item Ilma materjale läbimata ei ole palju mõtet kohal olla
		\end{itemize}
		\item Slaidid tulevad üsna pealiskaudsed
		\item Pidage syllabuse dokumendil silma peal: sellest tuleb kogu aeg uusi versioone
	\end{itemize}
\end{frame}

\begin{frame}[standout]
Küsimusi loetu kohta?
\end{frame}


\begin{frame}{Filosoofia}
	\begin{itemize}
		\item Arhitektuur koosneb laias laastus järgmistest osadest
		\begin{description}
			\item[Teadmised] Universaalsed  faktid ja tunnustatud teooriad
			\item[Oskused] Võimekus kasutada konkreetseid tehnilisi ja mittetehnilisi vahendeid väärtuse loomiseks 
			\item[Kogemus] Arhitekti isiklik (elu)kogemus, anne ja stiil
		\end{description}
		\item Keskendume esimesele andes minimaalsed vajalikud oskused teadmise rakendamiseks
		\begin{itemize}
			\item Ei räägi TOGAFist, COBITist, ITIList
			\item Konteineritest ja NoSQList ka mitte
			\item Aga räägin väga palju \enquote{pehmetest} asjadest
		\end{itemize}
		\item Pigem vähem aga üsna põhjalikult
	\end{itemize}
\end{frame}

\begin{frame}{Peamised autoriteedid}
	\begin{description}
		\item[Akadeemikud] Crawley, de Weck, Cameron ja teised MIT \emph{Systems*} inimesed
		\item[Arhitektuuripraktikud] Martin Fowler ja Thoughtworks
		\item[Tarkvarapraktikud] Fred Brooks, Joel Spolsky, Kent Beck etc.
\end{description}
\end{frame}

\begin{frame}[fragile]
	\begin{center}
		\LARGE{\textbf{See kõik erineb oluliselt teistest EA käsitlustest Eesti ülikoolides}}
		\\[4cm]
		\small{Mõni ei ole nõus, mõni ei saa aru, mõni kardab.\\ Ma respekteerin teie eriarvamust, \\teie palun respekteerige meie kõigi aega}
	\end{center}
\end{frame}


\begin{frame}[standout]
Kes te olete ja mida te siit ootate?
\end{frame}

\section{Põhimõisted ja akadeemiline kontekst}
\begin{frame}{Mis on arhitektuur?}
	\begin{itemize}
		\item Ühest definitsiooni ei ole
		\begin{itemize}
			\item \enquote{Süsteemi elementide ja nendevaheliste seoste kirjeldus}
			\item \enquote{Fundamental organization of a system, embodied in its components, their relationships to each other and the environment, and the principles governing its design and evolution}
			\item \enquote{The embodiment of concept, and the allocation of physical/informational function to elements of form and definition of structural interfaces among the objects}
			\item \enquote{Mida iganes programmeerija ei suuda teha}
		\end{itemize}
		\item Hoidume samuti ühest definitsiooni andmast kuid toetume paljus MIT omale
	\end{itemize}

\end{frame}

\begin{frame}{Mis on ettevõttearhitektuur?}
	\begin{itemize}
		\item Klassikalises vaates on kaks arhitekti
		\begin{itemize}	
			\item Ettevõttearhitekt, kes on nagu linnaplaneerija
			\item Lahendusearhitekt, kes on nagu maja arhitekt
		\end{itemize}
		\item Aga kui on hästi suur maja ja väga väike linn?
		\item Aga äkki on väga suur linn väga pisikeste majadega?
		\item Aga kui meil on suur linn paljude linnaosadega?
	\end{itemize}
\end{frame}

\begin{frame}{Süsteem ja selle piirid}
Süsteem on \enquote{kogum olemeid ja nendevahelisi suhteid, millede funktsionaalsus on suurem, kui üksikute olemite funktsionaalsuste summa}
	\begin{itemize}
		\item Millised (ja mis tüüpi) olemid on süsteemis sees ja millised väljas?
		\item Misasi on olem ja kuidas ta erineb allsüsteemist? 
		\item Kuidas käsitleda süsteemi piiril asuvaid olemeid ehk liideseid?
	\end{itemize}
\end{frame}

\begin{frame}[fragile]
	\begin{center}
		\LARGE{\textbf{Ettevõtte- ja lahendusearhitekti töö on olemuslikult sama}}
		\\[4cm]
		\small{Kuid vahendid võivad olla erinevad: \\mida kaugemale koodist seda rohkem lisandub mittetehnilisi osiseid}
	\end{center}
\end{frame}

\begin{frame}[fragile]
	\begin{center}
		\LARGE{\textbf{Käsitleme ettevõttearhitektina inimest, kes hoolitseb organisatsiooni kui sellise tehnoloogilise arhitektuuri eest}}
		\\[4cm]
		\small{Jah, \enquote{tehnoloogia} ja \enquote{kui sellise} on venivad mõisted. \\Tuletage meelde juttu süsteemi piiridest}
	\end{center}
\end{frame}

\begin{frame}{Mida arhitekt teeb?}
	Arhitekti\footnote{Või kelle iganes, kes seda rolli täidab} töö koosneb laias laastus järgmisest:
	\begin{enumerate}
		\item Võrdleme omavahel
		\begin{itemize}
			\item Süsteemi olek
			\item Implitsiitne või eksplitsiitne ärivajadus
			\item Strateegiline ideaal
		\end{itemize}
		\item Muudatuse planeerimine
		\item Muutuse toetamine
		\item GOTO 1
	\end{enumerate}
	\begin{center}
			\textbf{Mis süsteemist on jutt?}
	\end{center}
\end{frame}

\begin{frame}{Miks arhitekt seda kõike teeb?}
	\begin{itemize}
		\item Miks inimene üldse just sellel tööl käib?
		\item Arhitekt võib luua väärtust läbi
		\begin{itemize}
			\item keerukuse juhtimise
			\item muudatuste võimaldamise
			\item teadmuse jagamise
			\item strateegia toetamise
		\end{itemize}
		\item Aga kindlasti mitte läbi \enquote{EAga tegelemise}
		\begin{itemize}
			\item Saba ei tohi koera liputada
			\item Artefaktide tootmine, disainide kinnitamine vms. ei ole asi iseeneses
		\end{itemize}
	\end{itemize}
	\begin{center}
		\textbf{Arhitekti roll lisab väärtust vaid läbi teiste rollide}
	\end{center}
\end{frame}

\section{Harjutus}
\begin{frame}[fragile]
	\begin{center}
		\LARGE{\textbf{Mida teeb ettevõttearhitekt Microsoftis?}}
		\\[4cm]
		\small{Microsoft kui hiiglaslik rahvusvaheline tarkvaraettevõte}
	\end{center}
\end{frame}

\begin{frame}[fragile]
	\begin{center}
		\LARGE{\textbf{Mida teeb ettevõttearhitekt Eesti Energias?}}
		\\[4cm]
		\small{Eesti Energia kui suur tehnoloogiaettevõte}
	\end{center}
\end{frame}

\begin{frame}[fragile]
	\begin{center}
		\LARGE{\textbf{Mida teeb ettevõttearhitekt Swedbank Baltic'us?}}
		\\[4cm]
		\small{Swedbank Baltic kui rahvusvaheline osa suurest rahvusvahelisest mitte-tehnoloogiaettevõttest}
	\end{center}
\end{frame}

\begin{frame}[fragile]
	\begin{center}
		\LARGE{\textbf{Mida teeb ettevõttearhitekt RIAs?}}
		\\[4cm]
		\small{Riigi Infosüsteemi Amet kui taristut pakkuv riigiasutus}
	\end{center}
\end{frame}

\begin{frame}[fragile]
	\begin{center}
		\LARGE{\textbf{Mida teeb ettevõttearhitekt Transferwise's?}}
		\\[4cm]
		\small{Transferwise kui kasvufaasi iduettevõte}
	\end{center}
\end{frame}

\begin{frame}{Harjutuse kokkuvõte}
	\begin{itemize}
		\item Milles rollid erinevad?
		\item Milles rollid sarnanevad?
		\item Kas rollid lähevad kokku räägitu ja loetuga?
	\end{itemize}
\end{frame}

\begin{frame}{Kordame}
	\begin{itemize}
		\item Arhitektuur
		\item Ettevõttearhitekt ja arhitekt
		\item Arhitekti töö sisu
	\end{itemize}
\end{frame}

\begin{frame}{Järgmine kord}
\begin{itemize}
	\item Keskendume kommunikatsioonile: arhitekt peab suutma oma mõtteid ajas ja ruumis edasi anda
	\item Harjutame praktilist modelleerimist ja puudutame kommunikatsiooniteooria põhitõdesid
	\item Räägime põgusalt sellest \emph{mida} kommunikeerida
	\item Kirjandus
	\begin{itemize}
		\item Fowleri UMLi raamat tasub endale hankida, see on väga abiks
		\item Proovige ORMi pärast de Wecki slaidide lugemist: ta tahab natuke harjumist
	\end{itemize}
	\end{itemize}
\end{frame}
\begin{frame}[standout]
Küsimusi?
\end{frame}

\end{document}
