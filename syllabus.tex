\documentclass[nobib]{tufte-handout}
\usepackage[utf8]{inputenc}
\usepackage[estonian]{babel}
\usepackage[style=english]{csquotes}

\title{IDU1321 - Ettevõtte äriarhitektuur syllabus}
\author[Andres Kütt]{Andres Kütt}

%\date{04.08.2017}
%\geometry{showframe} % debug

\usepackage{natbib}

\usepackage{graphicx}
 \setkeys{Gin}{width=\linewidth,totalheight=\textheight,keepaspectratio}
  \graphicspath{{graphics/}} % set of paths to search for images
\usepackage{amsmath}  % extended mathematics
\usepackage{booktabs} % book-quality tables
\usepackage{units}    % non-stacked fractions and better unit spacing
\usepackage{multicol} % multiple column layout facilities
\usepackage{lipsum}   % filler text
\usepackage{fancyvrb} % extended verbatim environments
  \fvset{fontsize=\normalsize}% default font size for fancy-verbatim environments

% Standardize command font styles and environments
\newcommand{\doccmd}[1]{\texttt{\textbackslash#1}}% command name -- adds backslash automatically
\newcommand{\docopt}[1]{\ensuremath{\langle}\textrm{\textit{#1}}\ensuremath{\rangle}}% optional command argument
\newcommand{\docarg}[1]{\textrm{\textit{#1}}}% (required) command argument
\newcommand{\docenv}[1]{\textsf{#1}}% environment name
\newcommand{\docpkg}[1]{\texttt{#1}}% package name
\newcommand{\doccls}[1]{\texttt{#1}}% document class name
\newcommand{\docclsopt}[1]{\texttt{#1}}% document class option name
\newenvironment{docspec}{\begin{quote}\noindent}{\end{quote}}% command specification environment

\begin{document}
\maketitle
\begin{abstract}
\noindent
Käesolev dokument on Tallinna Tehnikaülikooli aine ``IDU1321 - Ettevõtte äriarhitektuur`` ainekaardi laiendus sisaldades bibliograafiat, igakordset lugemismaterjali ning praktikumi ülesandeid. Tegu on sisulisema lisandusega akadeemilisele ainekaardile, mis iganes lahkhelide korral kehtib ametlik ainekaart. Paber hakkab aine edenedes täienema, pea versioone silmas!
\end{abstract}

\section{Versioonid}
\begin{description}
	\item[1.0] Esimene levitatav 
	\item[1.1] Lisatud lõik iseseisva töö kohta
	\item[1.5] Täiendatud teise loengu kirjanduse loetelu
\end{description}

\section{Meetod}
Õpitulemused saavutatakse tuginedes järgmistele põhimõtetele
\begin{itemize}
	\item Teema on lai, seega kraabime vaid selle pinda vähestest, praktliselt kasulikest, kohtadest
	\item Valdkond on abstraktne, seega räägime samast asjast mitu korda eri nurkade alt. Pigem vähem aga sügavamat teadmist
	\item Eelistame akadeemilist teadmist raamistikele ja muule kommertsteabele. Esimene on püsiv väärtus, teised pigem mitte
	\item Modelleerimine on kommunikatsioonivahend, seega suhtume asjakohastesse standarditesse loovalt
	\item Tegemist on inseneeriaga, seega õpime tulemusi praktiliselt rakendama\sidenote{Kuna aega on vähe, vaatame ühte lähenemist paljudest. See konkreetne on valitud, sest tal on praktilised väljundid korporatiivkommunikatsioonis}
\end{itemize}

\section{Õpiväljundid}
Kursuse läbinud tudeng 
\begin{enumerate}
	\item Tunneb ettevõtte arhitektuuri ning äriarhitektuuri peamise mõisteid ja kontseptsioone, saab aru nende seosest infosüsteemidega 
	\item Oskab suurest süsteemist (nagu ettevõte ja tema infosüsteem) tervikpilti luua ja struktuurselt esitada. 
	\item Oskab struktuurselt mõelda ja arutleda. 
	\item Oskab modelleerida ettevõtte põhiprotsesse ning dekomponeerida neid kasutuslugudeks, funktsionaalseteks komponentideks ja tehnoloogilisteks lahendusteks
	\item Tunneb ja oskab kasutada vähemalt ühte infosüsteemi strateegilise analüüsi ja ärimodelleerimise metoodikat. 
	\item Oskab kasutada UMLi ja ORMi strateegilise analüüsi põhimudelite ja -vaadete koostamiseks 
	\item Oskab täita äriarhitekti, ärianalüütiku ja äridisaineri rolle infosüsteemi strateegilise analüüsi projektides. 
	\item Teab keerukuse ja selle juhtimise põhimõisteid
	\item Oskab eri vaatepunktidest hinnata lihtsamate süsteemide keerukust
	\item Oskab luua ja analüüsida seoseid organisatsiooni, selle strateegia ning infosüsteemi vahel
\end{enumerate}

\section{Loengud/praktikumid}
Aine sisu on toodud tabelis \ref{tab:content}. Loengus ja praktikumis osalemise eelduseks on, et viidatud artiklid, ja presentatsioonid on läbi loetud\sidenote{Raamatute läbimist ei eelda, need on toodud teemaga süvitsi minna soovijatele}. Loengu eesmärk on loetut täiendada, vastata küsimustele ning siduda eri allikaid. Kirjandust loengus ümber ei jutustata, edukas eksam eeldab sellega tutvumist. 

Loeng koosneb kolmest osast:
\begin{enumerate}
	\item Käime üle peamised ideed, mõisted ja kontseptsioonid
	\item Tegeleme kas kõik koos või väiksemates rühmades harjutustega
	\item Kokkuvõte harjutustest, sissejuhatus järgmisse teemasse ja ülevaade kirjandusest
\end{enumerate}

\begin{table*}[ht]
\small
	\centering
	\fontfamily{ppl}\selectfont
	\begin{tabular}{lp{5cm}p{6cm}p{4cm}}
		\toprule
		& Loeng & Harjutus & Kirjandus\\
		\midrule
		1 & Aine struktuur ja akadeemiline kontekst. EA ja ümberkaudsed mõisted. Ettevõttearhitekti roll. Eri vaated ettevõttele & Mida teeb ettevõttearhitekt Microsoftis, Eesti Energias, Swedpankis, Riigi Infosüsteemi Ametis, Transferwise's?  & \cite{parsons2005enterprise}; \cite{hickey}; \cite{sysengineering}; \cite{winter2006essential}\\
		2 & Modelleerimine. Kasud, kahjud. Numbrilised meetodid. UMLi ja ORMi põhitõed. Staatilised ja dünaamilised vaated. & Modelleerime: ERP, veebipood, kohtvõrk, EMTA, Avaandmed. Loengu pidamise äriprotsess & \cite{OPM}; \cite{umldistilled}, \cite{heumann2005introduction}\\
		3 & Arhitektuur, väärtus ja äri& & \\
		4 & Äriprotsess ja selle modelleerimine & & \cite{wohed2006suitability}\\
		5 & Äriprotsessist infosüsteemini. Linnaplaneerimine & & \cite{longepe2003enterprise}; \cite{bertin2014urbanization}\\
		6 & Infosüsteemist äriprotsessini. Infosüsteemi mõju organisatsioonile. Conway seadus, selle implikatsioonid & & \cite{conway1968committees}; \cite{maccormack2012exploring}\\
		7 & Süsteemiarhitektuuri alused & & \cite{crawley2015system}\\
		8 & Keerukus ja selle juhtimine & & \cite{holt2017so}\\
	\end{tabular}
	\caption{Loengute teemad, ülesanded ja kirjandus}
	\label{tab:content}
\end{table*}

\section{Iseseisev töö}
Ainekaardilt: \enquote{Ülesandeks on välja töötada konkreetse (reaalse või väljamõeldud) organisatsiooni infosüsteemi viiekihiline mudel. Lähtudes konkreetsest äriprotsessist koostatakse organisatsiooni, kasutuslugude, funktsionaalsete osiste ja tehniliste komponentide mudelid. Tulemus kantakse ette.}

\section{Loengud}
\subsection{Esimene}

Kordamisküsimused kirjanduse tarbeks on järgmised:
\begin{itemize}
	\item Mis on ettevõttearhitekti\sidenote{Parema puudumisel kasutame siin ja edaspidi toortõlget ingliskeelsest \emph{Enterprise Architect}i mõistest} roll \citeauthor{parsons2005enterprise}, \citeauthor{sysengineering} ja \citeauthor{winter2006essential} arvates, mis on peamised erinevused?
	\item Mis on peamised EA kihid \cite{winter2006essential} järgi?
	\item Millistest kuus vaadet eksisterivad arhitektuurile \citeauthor{sysengineering} järgi? Millisesse neist kuuluvad ORM ja UML?
	\item Mis on \citeauthor{hickey} järgi peamine EA probleem agiilses transformatsioonis ning mis on peamised viisid selle ületamiseks?
\end{itemize}

Harjutuseks arutame koos ettevõttearhitekti rolli üle eri tüüpi ettevõtetes. Rolli põhiküsimused on:
\begin{itemize}
	\item Mis on lahendatav põhiprobleem?
	\item Kellele raporteeritakse?\sidenote{Laiemalt, kus organisatsioonis roll paikneb. Pane tähele, kuidas roll sõltub organisatsiooni struktuurist}
	\item Millised on õigused ja kohustused?
	\item Mis on kontrollitava süsteemi ulatus?
	\item Kes on otsesed ja kaudsed alluvad?
\end{itemize}

Vaadeldavad organisatsioonid:
\begin{itemize}
	\item Microsoft kui hiiglaslik rahvusvaheline tarkvaraettevõte
	\item Eesti Energia kui suur tehnoloogiaettevõte
	\item Swedbank Baltic kui rahvusvaheline osa suurest rahvusvahelisest mitte-tehnoloogiaettevõttest
	\item Riigi Infosüsteemi Amet kui taristut pakkuv riigiasutus
	\item Transferwise kui kasvufaasi iduettevõte\sidenote{Pane tähele, kuidas tegu on hübriidiga finants- ja tehnoloogiaettevõttest}
\end{itemize}

\section{Lisalugemist}
\begin{itemize}
	\item \cite{simon1996sciences}. Huvitav ja suhteliselt populaarne käsitlus keerulistest süsteemidest 
	\item  \cite{stanford2005guide}.Economisti põhjalik vaade organisatsiooni kui sellise disaini
	\item \cite{fatolahi2006investigation}. Üsna põhjalik käsitlus Zachmani EA raamistikust ja selle rakendamisest UMLi abil. Hulgaliselt kasulikke viiteid, päris hea sissevaade akadeemilisse modelleerimsie ja EA maailma
\end{itemize}

\bibliography{idu1321}
\bibliographystyle{plainnat}
\end{document}